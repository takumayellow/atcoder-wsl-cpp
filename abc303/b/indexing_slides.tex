\documentclass[dvipdfmx,aspectratio=169]{beamer}
\usepackage{amsmath}
\usepackage{booktabs}
\usepackage{pxjahyper}

\usetheme{Madrid}
\usecolortheme{default}
\setbeamertemplate{navigation symbols}{}

\title{ABC303 B\\\small \texttt{x[j]--} を図で理解する}
\author{}
\date{}

\begin{document}

\begin{frame}
  \titlepage
\end{frame}

\begin{frame}{この資料のゴール}
  \begin{itemize}
    \item 入力が 1-index,配列添字が 0-index というズレを理解する.
    \item \texttt{x[j]--} が何をしているかを具体例で理解する.
    \item なぜ変換しないと危険か(範囲外アクセス)を理解する.
  \end{itemize}
\end{frame}

\begin{frame}{1-index と 0-index の対応}
  問題文の人番号(1-index)と,配列添字(0-index)は次のように対応します.

  \vspace{0.6em}
  \begin{center}
    \begin{tabular}{ccccc}
      \toprule
      人番号(入力) & 1 & 2 & 3 & 4 \\
      \midrule
      添字(配列) & 0 & 1 & 2 & 3 \\
      \bottomrule
    \end{tabular}
  \end{center}

  \vspace{0.6em}
  つまり「各値を 1 減らす」と対応が取れます.
\end{frame}

\begin{frame}[fragile]{\texttt{x[j]--} は j 番目の値だけを 1 減らす}
\begin{verbatim}
for (int j = 0; j < n; j++) {
    cin >> x[j];   // 例: 1 3 2 4
    x[j]--;        // 各要素を1減らして0-index化
}
\end{verbatim}

  例として,入力が \texttt{1 3 2 4} の場合:

  \begin{center}
    \begin{tabular}{cccc}
      \toprule
      j & 読み込み直後 x[j] & \texttt{x[j]--} 後 x[j] & 備考 \\
      \midrule
      0 & 1 & 0 & 1番の人 $\rightarrow$ 添字0 \\
      1 & 3 & 2 & 3番の人 $\rightarrow$ 添字2 \\
      2 & 2 & 1 & 2番の人 $\rightarrow$ 添字1 \\
      3 & 4 & 3 & 4番の人 $\rightarrow$ 添字3 \\
      \bottomrule
    \end{tabular}
  \end{center}

  最終的に \texttt{x = [0, 2, 1, 3]} になります.
\end{frame}

\begin{frame}[fragile]{なぜ必要か: \texttt{adjacent[a][b]} は 0-index 前提}
\begin{columns}[T,onlytextwidth]
  \column{0.48\textwidth}
  \textbf{変換あり(正しい)}
  \begin{itemize}
    \item \texttt{x = [0,2,1,3]}
    \item 隣接ペア:
    \texttt{(0,2), (2,1), (1,3)}
    \item \texttt{adjacent[a][b]} の $a,b$ は
    すべて \texttt{0..n-1} に収まる.
  \end{itemize}

  \column{0.48\textwidth}
  \textbf{変換なし(危険)}
  \begin{itemize}
    \item \texttt{x = [1,3,2,4]} のまま
    \item 例えば \texttt{adjacent[2][4]} が出る
    \item \texttt{n=4} なら有効添字は \texttt{0..3}
    \item \texttt{4} は範囲外アクセス.
  \end{itemize}
\end{columns}
\end{frame}

\begin{frame}[fragile]{よくある誤解}
  \textbf{「配列に入れたから 1-index になる」わけではありません.}

  \begin{itemize}
    \item 1-index / 0-index は \textbf{入力仕様} で決まる.
    \item C++ の \texttt{vector} 添字は常に 0-index.
    \item 入力が \texttt{1..n} なら \texttt{-1} 変換が必要.
    \item 入力が最初から \texttt{0..n-1} なら不要.
  \end{itemize}

\begin{verbatim}
// 入力仕様が0-indexの問題なら
cin >> x[j];
// x[j]--; は不要
\end{verbatim}
\end{frame}

\begin{frame}{確認問題}
  \textbf{Q.} \texttt{n=5},入力が \texttt{2 5 1 4 3} のとき,
  \texttt{x[j]--} 後の配列は?

  \vspace{1em}
  \textbf{A.} \texttt{1 4 0 3 2}

  \vspace{1em}
  これで \texttt{adjacent[a][b]} の添字は常に \texttt{0..4} に収まります.

  \vspace{1.2em}
  \begin{block}{覚えることは 1 行}
    入力が 1-index なら,配列で使う前に 0-index に合わせる.
  \end{block}
\end{frame}

\end{document}
